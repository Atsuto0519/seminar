\documentclass[12pt, aJ4]{jsarticle}

\usepackage{ascmac}
\usepackage{amsmath}
\usepackage{additional}

\begin{document}

\title{ゼミテンプレ}
\author{稲毛惇人}
\maketitle

\section{はじめに}
コメントアウトはCommand+\{,解除はCommand+\}.

\section{手法}
\begin{myframe}{パラメータ}
	\begin{tabular}{ll}
		$c$ & :状態数 \\
		$m$ & :出力記号の数 \\
		$s_{t} \in \{\omega_{1}, \omega_{2},..., \omega_{c}\}$ & :時点$t$での状態 \\
		$x_{t} \in \{v_{1}, v_{2},..., v_{m}\}$ & :時点$t$での観測結果(出力記号) \\
		$\mathbf{s} = s_{1}s_{2}\cdots s_{t} \cdots s_{n}$ & :状態系列 \\
		$\mathbf{x} = x_{1}x_{2}\cdots x_{t} \cdots x_{n}$ & :観測記号系列 \\
		$a_{ij},a(\omega_{i}, \omega_{j})$ & :状態$\omega_{i}$から状態$\omega_{j}$への遷移確率 \\
		$b_{jk},b(\omega_{j}, v_{k})$ & :状態$\omega_{j}$で$v_{k}$を出力する確率 \\
		$\pi_{i}$ & :初期状態($t=1$)が$\omega_{i}$である確率 \\
		$\bm{A}$ & :$a_{ij}$を($i,j$)成分としてもつ$c \times c$の行列 \\
		$\bm{B}$ & :$b_{jk}$を($j,k$)成分としてもつ$c \times m$の行列 \\
		$\bm{\pi}=(\pi_{1}, \pi_{2}, ..., \pi_{c})$ & :$\pi_{i}$を成分としてもつ$c$次元のベクトル
	\end{tabular}
\end{myframe}

% 参考文献
\begin{thebibliography}{9}
	\renewcommand{\baselinestretch}{1.0}
	\small
	\bibitem{lit:Tmp}
		著者名, 
		``作品名'', 
		出版会社, 20XX.
\end{thebibliography}
		
\end{document}
